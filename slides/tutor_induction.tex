\documentclass[10pt]{beamer}

\usetheme[progressbar=frametitle]{metropolis}
\usepackage{appendixnumberbeamer}

\usepackage{booktabs}
\usepackage[scale=2]{ccicons}

\usepackage{pgfplots}
\usepgfplotslibrary{dateplot}

\usepackage{xspace}
\newcommand{\themename}{\textbf{\textsc{metropolis}}\xspace}

\usepackage{hyperref}
\hypersetup{
    colorlinks=true,
    linktoc=none,
    urlcolor=orange,
    }

\title{Tutoring in computer labs}
\subtitle{SoM tutor induction}
% \date{\today}
% \date{Wed 14th September 2022}
\author{Charlotte Desvages (charlotte.desvages@ed.ac.uk)}
\institute{School of Mathematics}
% \titlegraphic{\hfill\includegraphics[height=1.5cm]{logo.pdf}}

\begin{document}

\setbeamercovered{transparent}
\metroset{block=fill}

\maketitle

\begin{frame}{Overview}
  \setbeamertemplate{section in toc}[sections numbered]
  \tableofcontents%[hideallsubsections]
\end{frame}

\section{Computer labs in the School}

\begin{frame}[fragile]{Computer labs in SoM}

    \begin{itemize}[<+->]
        \item Most are \textbf{programming labs}, typically Python or R (with some exceptions).
        \item Main opportunity for students to \textbf{practice} and get help from classmates/tutors.
        \item Sometimes in computer lab room, sometimes in ``traditional'' workshop room (with students working on laptops).
        \item Guidance for tutoring workshops also applies.
    \end{itemize}

\end{frame}

\begin{frame}[fragile]{Questions to ask your Course Organiser}

    Different courses do computer labs differently, and for different purposes.
    \pause

    \begin{itemize}[<+->]
        \item \textbf{Learning outcomes} related to computing?
            \begin{itemize}
                \item general programming skills
                \item mathematical/statistical computing
                \item proficiency with a particular piece of software
                \item etc\ldots
            \end{itemize}
        \item \textbf{Software/platforms} students are expected to use in the course?
            \begin{itemize}
                \item Install/set up your own machine with the same workflow.
            \end{itemize}
        \item Are students expected to \textbf{collaborate} during the workshops?
            \begin{itemize}
                \item If so, how? (informally, in groups of 3+, pair programming\ldots)
                \item If tasks are assessed, what is acceptable collaboration?
            \end{itemize}
        \item \textbf{Marking and feedback:} what, how, and when?
        \item \textbf{Generative AI:} is it allowed? encouraged? in what situations?
    \end{itemize}

  
\end{frame}

\section{Preparing a computer workshop}

\begin{frame}[fragile]{Preparation}
    \begin{itemize}[<+->]
        \item CO will provide task and solutions in advance.
        \item \textbf{Try} the task yourself \textbf{without looking at solutions}.
            \begin{itemize}
                \item Anticipate different ways that students might think about the task, and where they might get stuck.
                \item Make notes of any useful bits of lecture notes, software documentation, or previous weeks' exercises to refer to.
            \end{itemize}
        \item Ideally, keep the model solution only to check that results are correct. Students come up with lots of creative ways to solve a task -- meet them where they are!
    \end{itemize}
\end{frame}

\section{Helping students during labs}

\begin{frame}{Helping students during labs}
    Same principles as for maths workshops:
    \begin{itemize}
        \item Don't give away the solution.
        \item Ask students to explain their thinking to you and to each other.
        \item Give pointers to course materials, documentation, etc.
    \end{itemize}
\end{frame}

\begin{frame}{Common questions and problems}
    \setbeamercovered{invisible}
    \begin{itemize}
        \item ``I don't know where to start\ldots''
        \item ``What does this function do?'' \onslide<3->{\alert{I don't know, let's find out!}}
        \item ``Is this correct/will this work?'' \onslide<4->{\alert{I don't know, let's find out!}}
        \item ``This is not working.''
    \end{itemize}
    \pause
    Two of these questions are easy to answer\ldots
    \setbeamercovered{transparent}
\end{frame}

\begin{frame}{I don't know where to start!}
    Translating problem to code is a skill that needs practice!
    \pause

    A useful guide: \href{https://adhilton.pratt.duke.edu/sites/adhilton.pratt.duke.edu/files/u37/iticse-7steps.pdf}{the \alert{Seven Steps} method}.
    \begin{itemize}[<+->]
        \item Get some \textbf{pen and paper}.
        \item Work out \textbf{an example by hand}.
        \item \textbf{Retrace your steps}: write down exactly what you've done.
        \item \textbf{Generalise} the steps above to arbitrary values.
        \item \textbf{Test} your procedure on a \text{different} example.
        \item Then, and \alert{only} then, translate to code.
        \item Test your programme on more examples.
    \end{itemize}
\end{frame}

\begin{frame}{This is not working\ldots}
    \metroset{block=fill}
    \begin{exampleblock}{Teach a man to fish\ldots}
        Tutoring is \textbf{not} debugging students' code for them -- it's helping them to develop the right habits to troubleshoot their own problems.

        Computer labs can be the only place and time for students to learn to find and fix bugs, with guidance and support from tutors.
    \end{exampleblock}
    \pause
    \textbf{Resist the temptation to take the keyboard away from a student!}
    
    \begin{itemize}
        \item Student won't remember what you've done.
        \item Very easy to use keyboard shortcuts without the student noticing.
        \item Very easy to skip explaining steps because we are used to them.
        \item \alert{\emph{Very easy to accidentally convince a student that debugging requires expert knowledge that they do not have.}}
    \end{itemize}
\end{frame}

\begin{frame}{This is not working\ldots}
    Even though they provide incredibly useful information, novice coders are often fearful of \textbf{error messages}.

    \begin{itemize}[<+->]
        \item As soon as a student comes to you with a runtime error: ``let's look at the error message.''
        \item Help them interpret the information there:
            \begin{itemize}
                \item \textbf{Where} is the error in your code?
                \item What \textbf{type} of error is this? Do you remember seeing an error like this before?
                \item Googling an obscure error message can be helpful!
            \end{itemize}
    \end{itemize}

    \pause
    \textbf{Demystifying} errors is important to give students confidence. It's absolutely fine (I'd even say, encouraged) to say ``I don't know'', to ask for help from other tutors or students in the room, to run broken code, to spend a lot of time tracking down a bug with a student.
\end{frame}

\begin{frame}{Practical strategies: the rubber duck}
    For structure or logical issues: \href{https://rubberduckdebugging.com/}{\alert{Rubber duck debugging}}
    \pause

    \begin{exampleblock}{Become the rubber duck.}
        Ask students to explain to you, \textbf{line by line, in excruciating detail,} what their code is doing.

        In the large majority of cases, they will find their mistake as they're explaining it.
    \end{exampleblock}

    \pause
    As a tutor, you can be a slightly more active rubber duck.
    \begin{itemize}[<+->]
        \item \textbf{Ask prompting questions} to help students through the explanation. (Are you sure? How does that function work? How many times do you do this? Let's try an example; etc.)
        \item Encourage them to \textbf{Google things}, and to reuse code snippets responsibly. (Ask me later about citing code appropriately!)
        \item Encourage them to \textbf{display} intermediate values, or plot results, to check that their code is actually doing what they intend.
    \end{itemize}
\end{frame}

\section{Putting it into practice}

\section{Resources}

\begin{frame}{Resources}
    \begin{itemize}
        \item The Seven Steps method: \href{http://adhilton.pratt.duke.edu/sites/adhilton.pratt.duke.edu/files/u37/iticse-7steps.pdf}{poster (linked above)} and accompanying \href{https://dl.acm.org/doi/10.1145/3300115.3309508}{paper} with further details.
        \item The \href{https://teachcomputing.org/pedagogy}{Teach Computing} project has a good collection of resources. Highlights on 2 resources to support program comprehension:
            \begin{itemize}
                \item \href{https://blog.teachcomputing.org/code-tracing/}{Code tracing}
                \item The \href{https://blog.teachcomputing.org/quick-read-understanding-program-comprehension-using-the-block-model/}{block model}
            \end{itemize}
        \item Brown, N., \& Wilson, G. (2018). Ten quick tips for teaching programming. PLoS computational biology, 14(4), e1006023. \url{https://doi.org/10.1371/journal.pcbi.1006023}
        \item The \href{https://computinged.wordpress.com/about/}{Computing Education Research Blog} by Prof. Mark Guzdial at the University of Michigan.
        \item Software Carpentry's \href{https://carpentries.github.io/instructor-training/}{instructor training material}. Includes evidence-based, practical advice on supporting students in learning computing.
    \end{itemize}
\end{frame}

\end{document}
